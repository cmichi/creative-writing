%!TEX TS-program = xelatex
%!TEX encoding = UTF-8 Unicode

\documentclass[11pt,paper=a5,pagesize,english,openright,twoside]{scrbook}

% switch margins
% http://tex.stackexchange.com/questions/10128/two-sided-document-reverse-page-margins-for-hardcopy
% http://stackoverflow.com/questions/2565755/latex-book-class-twosided-document-with-wrong-margins
%\let\tmp\oddsidemargin
%\let\oddsidemargin\evensidemargin
%\let\evensidemargin\tmp
%\reversemarginpar

\usepackage{typearea}
\typearea[10mm]{10}

\usepackage{xunicode}
\usepackage{xltxtra}
\usepackage{verbatim}

\defaultfontfeatures{Mapping=tex-text}
\usepackage{fontspec, graphicx}
\usepackage{lettrine}
\usepackage{setspace} 
\usepackage{afterpage} 
\usepackage{floatpag}
\usepackage{ragged2e}

\usepackage{multicol}

\usepackage{textcomp}

\usepackage{fancyhdr}
\pagestyle{fancy}
\fancyfoot{} % clear all footer fields
\fancyfoot[LE,RO]{\centering{\thepage}}
\renewcommand{\headrulewidth}{0.0pt}
\renewcommand{\footrulewidth}{0.0pt}

%\setmainfont[Ligatures = {Common, Rare}]{Adobe Garamond Pro}
\setmainfont[Ligatures = {Common, Rare}]{PF Din Text Cond Pro}
\parindent0mm
\parskip3mm

\usepackage{polyglossia}
\setmainlanguage[variant=british]{english}

\newcommand{\engine}[0]{ {\fontspec{Latin Modern Roman}\normalsize \XeLaTeX}}
\newcommand{\enginetwo}[0]{ {\fontspec{Latin Modern Roman}\normalsize \LaTeX}}
\newcommand{\pc}[1]{{\fontspec{OCRA}\scriptsize \addfontfeature{LetterSpace=-12.0,WordSpace=.7}{#1}}}
\newcommand{\q}[1]{»#1«}
\newcommand{\newdoublepage}{\newpage\thispagestyle{empty} \ \newpage \thispagestyle{empty}}

%\usepackage{geometry}

% for centered fixed-width columns in tabular environment
% http://tex.stackexchange.com/questions/12703/how-to-create-fixed-width-table-columns-with-text-raggedright-centered-raggedlef
\usepackage{array}
\newcolumntype{L}[1]{>{\raggedright\let\newline\\\arraybackslash\hspace{0pt}}m{#1}}
\newcolumntype{C}[1]{>{\centering\let\newline\\\arraybackslash\hspace{0pt}}m{#1}}
\newcolumntype{R}[1]{>{\raggedleft\let\newline\\\arraybackslash\hspace{0pt}}m{#1}}


\begin{document}

%\ 
%\thispagestyle{empty}
%\newpage
%\ 

%\newgeometry{margin=1.3cm}
\thispagestyle{empty}
\raggedbottom

\vspace{6.0cm}
\ \\ 

%\begin{minipage}[t]{7cm}
\begin{addmargin}[-1.10cm]{0cm}
{\fontspec{Gotham Black}\small
	\begin{spacing}{15}
	\noindent
	\fontsize{96pt}{8pt}\selectfont
		\hspace{.0cm}{\addfontfeature{LetterSpace=29}DYST}\\
		{\addfontfeature{LetterSpace=33}OPIA}\\
		\mbox{\addfontfeature{LetterSpace=10.0}TO\hspace{1.25cm}GO}\\
	\end{spacing}
}
%\end{minipage}
\end{addmargin}

%\restoregeometry

\thispagestyle{empty}
\newpage
\

\thispagestyle{empty}
\newpage

\thispagestyle{empty}
\afterpage{
\thispagestyle{empty}

\begin{figure*}
\thispagestyle{empty}
\floatpagestyle{empty}

%\begin{addmargin}[7em]{2em}
\setstretch{2.0}
\begin{center}
\textit{A long, long time ago\\
I can still remember\\
how that music\\ 
used to make me smile\\
%And I knew if I had my chance that I could make those people dance\\
%And maybe they'd be happy for a while.
}
\end{center}
%\end{addmargin}

\begin{center}\scriptsize
Don McLean --- American Pie
\end{center}

\clearpage
\floatpagestyle{empty}
\end{figure*}

\thispagestyle{empty}
\clearpage
}

\newpage \ \newpage

\thispagestyle{empty}
\ \newpage


\thispagestyle{empty}
%\newgeometry{margin=0cm}
\begin{figure*}
\floatpagestyle{empty}

\vspace*{3.0cm}
{\parindent0mm  

\center
	\hspace{0.0cm}
	\Large\textbf{The Day the Music died}\\
	\ \\
	\ \\
	\ \\
	\ \\
	%\large\textsc{--Michael Mueller--}\\
}
\ \\
\end{figure*}

%\restoregeometry


\pagebreak
\clearpage
\
\thispagestyle{empty}
\newpage
%\restoregeometry


\setcounter{page}{1}
\frenchspacing
\lettrine[lines=3,lhang=0.21,nindent=0em,findent=0.1em]{T}{he}
%Through the NSA leaks in the last year it has become clear that a lot of
%the dystopian predictions have become true. This is why I think it is time
%for a new dystopia. This is my proposition.
first time I heard music was at my eighteenth birthday. My parents had
saved a lot of money to afford me this gift: a ten second excerpt from 
Beethoven's \textit{Symphony No. 9}. They chose a sequence with the chorus 
singing a part of the Friedrich Schiller poem \textit{Ode to Joy}.

When I heard the music I was alone in my room and wow! I can tell you,
this blew me away. I was overwhelmed with emotions. I couldn't sleep for a 
whole night. Always trying to recall the notes from memory. This happened 
about forty years ago and since then a lot has changed. 

By the year 2100 technology had progressed so much that through advances in
electronic manufacturing it was now possible to replace the entire hearing 
organ through a much more capable implant. Such implants enabled all kinds 
of improvements: one could now listen directly to audio files or radio without 
any loss in quality due to analog signal transfer. One could also just switch 
the device off and eliminate any environmental noise --- no need for earplugs 
anymore! 
But most importantly, this led to the realization of a long awaited human 
dream: the chip had the capability to directly translate any foreign language 
into the individuals mother tongue. And all of that in real-time! There was 
no need to learn any new languages anymore. 
This was the breakthrough for the implants. In less than twenty years it was 
a very common surgery to substitute one's hearing organ with an electronic
chip. Such a surgery would usually be performed on three-year olds, since
it turned out to work best at that age. 

Sure, at first there was a small percentage of the population who refused to 
participate in this surgery.
It didn't take long until people who refused the implants were commonly 
considered somehow disabled. They even had to declare this ``disability'' 
in their passport, for others to recognize that they needed to be handled 
with special care. It took another
decade for the implant manufacturers to find out how much profit they could
make from having successfully privatized hearing. At first it was just a
marketing gag: certain bands released their music exclusively
under the hearing implants of certain manufacturers. Through legal measures
the manufacturers of other hearing implants were forbidden to play back these
songs.

Of course there were some manufacturers who consciously or unconsciously
violated these licensing terms. But after several major court cases
a lot of fines had to be paid for such violations. From then on the 
the devices started filtering, making only licensed and permitted signals 
audible.
The court cases had started a rat race amongst manufacturers: they each made
exclusive contracts with the record companies. In the end each individual
had only a very limited choice of music. This was why after several years
of hassle the manufacturers sat together to search for a better solution.
In the end they came up with an \textsc{all music flatrate\texttrademark}. 
To cover the licensing fees they made this offer available for
a small monthly fee. To make the \textsc{all music flatrate\texttrademark} 
even more attractive, it was from then on only possible to listen to music at 
all once the flatrate was purchased. People were still able to play the 
guitar or whistle, but soon the companies also closed this last loophole 
and you just wouldn't hear the sound of the guitar or the sound produced by 
whistling.
At first the signals of police or emergency cars were mistakenly filtered out 
by the implants. But the technology got better and certain adoptions had to 
be made within society. Emergency car sirens were changed to arhythmic, 
disharmonious noise and telephones stopped ringing.

After a while, the price was slowly increased, until music was a thing subject 
only to premium customers with enough money. Then, after a while, music
became a luxury --- something consumed only by the rich upper class. Amongst
common people, music is now seen as somewhat decadent. Many people don't
even know someone who has ever heard music. They don't know what they are
missing.

Today is my 60th birthday. A long time has passed since I have become an
adult. Throughout my entire life I never again had the pleasure of
listening to music. Though I have often recalled those memories of my
eighteenth birthday. Whenever I felt down or became sentimental, I would
try to recall the notes from my memory and oftentimes I couldn't help my
eyes filling with tears. Listening to music was surely one of the most 
beautiful moments in my life.

Since my eighteenth birthday I started to save up money. I had this
one dream in my mind, this dream for which I would finally, someday, have 
enough money. Today is the day. I have finally saved up enough. 
Today I will finally spend it on my life's dream. 

Music wasn't the only thing that got lost, other senses were substituted as well. 
The companies limited other functionalities to premium customers as well.
Today I have finally saved up enough money. Today I will finally buy 
myself the license for seeing colors.


%Whilst regularily reading the news I am on a regular basis quite erstaunt
%how dystopian our world has become in some cases. The NSA leaks, patent
%wars, sponsoring tanks, .etc.
%So coming up with a fictional dystopia took some effort. But even for this
%story the technical advances are quite far --- farer than most people know.
% (cochlea)

\clearpage
\thispagestyle{empty}

\ \\
\vspace{8cm}\\
%\sepg{}

\newpage
\thispagestyle{empty}
\
\newpage
\thispagestyle{empty}
\
\newpage




%\thispagestyle{empty}
%\parindent0mm
%\pagestyle{empty}
%\raggedbottom
\thispagestyle{empty}
%\fancyfoot{}
%\fancyfoot[C]{
	%\centering
	%Michael M\"uller\\
	%\pc{http://micha.elmueller.net}
%}

%\thispagestyle{empty}

%\include{appendix}

\begin{multicols}{2} 
%\thispagestyle{empty}
\textbf{Background Info}\\ \ \\

This text was typeset using the \textsc{pf din text condensed}. 
The title page was set using \textsc{gotham} --- a nearly monospaced
font designed by Tobias Frere-Jones for the GQ magazine.\\

On the software side this document was typeset using \engine.
The sourcecode used to render this document (and a resulting 
\textsc{PDF}) is available here:\\ 
\pc{https://github.com/cmichi/\allowbreak{}creative-writing}.\\
%A \textsc{PDF} version of this document is available there as well.\\

I gave the story to two proof readers, both of whom returned helpful 
remarks. Thank you, Katja Rogers and Markus Schnalke.
%\hspace{-0.1cm} \pc{https://github.com/cmichi/\-reworking-typography}\hspace{-0.1cm}. 

\columnbreak
%\thispagestyle{empty}

\textbf{What to do with this document now?}\\

%What should you do with this document now? 
Of course you are free to do whatever you want with it, but I suggest 
you either keep it, pass it on or leave it at some place where others will 
likely find it and read it.\\

\begin{center}
\vspace{-0.5cm}
	\includegraphics[height=1.0\baselineskip]{cc.eps}
	\includegraphics[height=1.0\baselineskip]{by.eps}
\end{center}

\vspace{-0.75cm}
This work is licensed freely under the \emph{Creative Commons Attribution 4.0 
International (CC BY 4.0) License}. To view a copy of this license, visit
\pc{http://creativecommons.org/\\licenses/by/4.0/}.\\ \ \\

\end{multicols}
\vfill
	\centering
	Michael M\"uller, July 2015\\
	\pc{http://micha.elmueller.net}
\thispagestyle{empty}


%\input{appendix}


%\newdoublepage
\newpage
\thispagestyle{empty}

%\hspace{-7.9cm}\textbf{Logbook}\\ \ \\
\begin{center}
\hspace{-1.1cm}\textbf{Logbook}
\end{center}
\ \\ \ \\

\begin{tabular}{C{0.0cm}C{0.20\textwidth} | C{0.26\textwidth} | C{0.38\textwidth}}
\hline
\vspace{0.3cm} & \hspace{-0.4cm}When? & Where? & Any thoughts?\\
\hline
& &\\
& &\\
& &\\
& &\\
& &\\
& &\\
& &\\
& &\\
& &\\
& &\\
& &\\
& &\\
& &\\
& &\\
& &\\
& &\\
& &\\
& &\\
& &\\
& &\\
& &\\
& &\\
& &\\
& &\\
\end{tabular}

%\newpage
%\thispagestyle{empty}
%\

\end{document}
